\chapter{\conclusionsname}
\label{chap:conclusions}

The increasing use of Internet of Things (IoT) devices and the resulting large amounts of data being produced present significant challenges for
cloud computing. While cloud computing has proven effective for many applications, it may not always be suitable for real-time constraints and
handling data from IoT devices.

Fog computing offers a promising solution by providing a computing model that sits between IoT devices and the cloud,
allowing for the collection, aggregation, and processing of data using a hierarchy of computing power. Combining fog computing with the cloud can
reduce data transfers and communication bottlenecks, as well as contribute to reduced latencies. However, realizing systems that operate in the
edge-cloud continuum is a complex challenge due to the heterogeneity of devices and dynamic requirements of today's systems.

Various approaches have been proposed to address these challenges, including self-organizing systems and methodologies such as osmotic computing and
DR-BIP. The orchestration of distributed applications requires careful management of the underlying infrastructure to enable the reuse of design
elements across different scenarios.

The thesis focuses more specifically on the pulverization approach: a framework that breaks the system behaviour into small computational pieces that
are continuously executed and scheduled in the available infrastructure.
In this way, the business logic of a system is neatly separated from infrastructure or deployment concerns enabling the concept of
\emph{deployment independent} systems.
Reuse and independency from the deployment are the two main pillars of the pulverization approach, by which the framework aims to enable the
deployment of systems in the edge-cloud continuum.

The main contribution of the thesis is the development of a framework that leverages the pulverization approach to deploy Cyber-Physical Systems.
The framework aims to lay the groundwork for closing the gap between the simulation of these systems and their deployment by exploiting the
pulverization methodology.

The framework is built trying to maintain ease of use, modularity, and extensibility by modeling well the foundational concepts of pulverization over
which the framework can be extended and improved.

Several technology solutions were examined that could support cross-platform targets allowing the framework to be used seamlessly across different
platforms and architectures. Kotlin multiplatform was identified as a suitable technology for the development of the framework since it has a wide
range of supported architectures.

The most significant implementation details and technology challenges that were faced during the development of the framework and what solutions were
employed to achieve it were reported.

Finally, topics such as testing and validation were covered by showing what strategies were used to validate the operation of the framework, as well
as significant demos were developed to show how the framework works in different contexts, each of which brings with it peculiar characteristics that
go to corroborate the proper operation and effectiveness of the framework.

From using the framework, it seemed evident that deployment aspects never appear during system implementation, leading to the advantage of focusing
most efforts on development, delegating infrastructure and communication aspects to the framework. In addition, it was apparent how the reusability
of the developed components is easily achieved by design: this leads to the same component being able to be reused in different deployment strategies
in different infrastructures, making it flexible to changes in how it is deployed.

\section{Future Works}
\label{sec:future-works}

The framework is still in its early stages of development and there is still a lot of work to be done to make it more robust and complete. The
following are some of the topics that could be explored in future work.

Due to the heterogeneity of the devices that can be used in the edge-cloud continuum, the framework should be able to support different
communication protocols. This would allow the framework to be used in a wider range of scenarios allowing the use of different communication
protocols based on device capabilities. Moreover, this work can be extended to support different communication protocols at the same time to
opportunistically exploit the best protocol for the actual system requirements or quality of services.

Dynamics is a key aspect of the edge-cloud continuum, and the framework should be able to support dynamic changes in the system. This would allow
the framework to be used in scenarios where the system is subject to changes in the number of devices, the number of resources, or opportunistically
exploit the best deployment strategy for the actual system.

Currently, to deploy the system, some manual steps such as containerizing the deployment units that will then run in the available infrastructure
must be performed. Automating this process can reduce the time to deploy the system by reducing the likelihood of error in container deployment.
As a consequence, it makes sense to study how downstream of containerization, containers can be automatically deployed into the infrastructure
through DevOps (CI/CD) methodologies.

\paragraph*{}

With this thesis, I brought in research topics and problems such as pulverization. It was motivating to delve into and understand the
concepts of pulverization and carry them into the implementation of a framework. Equally satisfying was seeing the framework work in different
contexts and understanding how it could be evolved in the future. This experience allowed me to study the literature and understand what
related work has already drawn useful insights from it to implement the framework.
In addition, this thesis allowed me to delve into the Kotlin ecosystem in its multiplatform version by understanding how this technology may be
suitable to support the implementation of the framework.

\todo{Vedere se aggiungere altro}
