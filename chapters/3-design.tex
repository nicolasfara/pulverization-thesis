\chapter{Design} % possible chapter for Projects
\label{chap:design}

This chapter describes the design choices and the overall architecture of the framework. \todo{Expand the introduction}

\section{Framework requirements}
\label{sec:framework-requirements}
\todo{Write the section about the project's requirements}

\section{Architectural design}
\label{sec:arch-design}

The framework is articulated in modules: each module takes into account a specific aspect of the pulverization.
The modularity of the framework enables from one side, the possibility to use only the needed modules, preventing the bloating of the project;
on the other side, modularity allows the customization of some implementations of the framework.

The two fundamentals modules of the pulverization framework are: \emph{core} and \emph{platform} which respectively defines the core concepts
of pulverization like the type of components and all the logic needed to run the pulverized system like defining the components reference,
loading the user-defined components and setup the communications between all of them.

The third module is \emph{rabbitmq-platform} which is highly dependent on the two modules described above and its purpose is to rely on
\textbf{RabbitMQ}\footnote{\textbf{RabbitMQ} is an open-source message-broker (or message-oriented-middleware) that originally implement
the \emph{AMQP} protocol and has since been extended with a plug-in architecture to support other protocols like \emph{MQTT}.} to enable
the communications between all the components.
This component manages all the low-level aspects related to communication like the connection to the broker, declaring queues and so on.

In~\Cref{fig:package-diagram} are represented all the framework's modules and the relationship between them.

\begin{figure}
    \centering
    \missingfigure[figwidth=\textwidth]{Package diagram showing the module relationship}
    \caption{Package diagram showing the modules that constitute the framework and their relationship.}
    \label{fig:package-diagram}
\end{figure}

The pulverization framework relies on a three-level architecture. Each level of the framework's architecture is designed to use the functionalities
of the layer above and makes accessible their functionalities to the layer below.

The described architecture takes with it the implicit ``one-way dependency'' where the layer below depends on the layer above and not vice versa.
The~\Cref{fig:framework-architecture} depicts the architecture's choice made to design and build the framework.

\begin{figure}
    \centering
    \missingfigure[figwidth=\textwidth]{Create a pyramid diagram showing the architecture of the framework emphasizing the grow-up dependency}
    \caption{Architectural diagram showing how the pulverization framework is designed.}
    \label{fig:framework-architecture}
\end{figure}

Since this is a framework, it will likely be used by several users; therefore, it is strategic to minimize the cognitive effort that the user will
have to make to use it.

For a framework to be successful and usable, it must be able to provide an incremental approach to its use,
meaning that it must provide only the abstractions necessary for its use while at the same time providing clarity in its innermost components so that
the user can understand how it works and possibly extend the framework with external modules.

The ``pyramid architecture'' used by the framework tries to apply the concept described above (\Cref{fig:pyramid-user-knowledge}):
the tip of the pyramid represents the components' abstraction defined by the pulverization and those components are used and built by the user.
Those abstraction needs to be as clear as possible from a software engineering perspective.
As you move down the pyramid, the complexity of the modules increases but the user's knowledge of them should decrease.

\begin{figure}
    \centering
    \missingfigure[figwidth=\textwidth]{pyramid architecture comparing the increase of complexity going down the pyramid and the decrease of user's knowledge going up the pyramid.}
    \caption{A correlation between the framework's complexity and the required user's knowledge to use the framework.}
    \label{fig:pyramid-user-knowledge}
\end{figure}

By designing the framework in this way, we open up different usage scenarios such as a basic use that requires only an understanding of the
basic concepts, to advanced uses that require an advanced understanding of the framework but enable its extension.

The sections below will describe the architectural choices made for each framework's module.

\subsection{Core module}
\label{sec:core-module}

\subsection{Platform module}
\label{sec:platform-module}

\subsection{Rabbitmq-platform module}
\label{sec:rabbitmq-platform-module}