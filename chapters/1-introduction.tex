\chapter{\introductionname}
\label{chap:introduction}
%----------------------------------------------------------------------------------------

With the Internet of Things more and more devices are connected to the network producing a very large amount of data. Cloud computing has been
established to acquire computational power and storage for various applications; however, it is not always suitable for handling a myriad of data
from IoT devices while respecting real-time constraints.
For these reasons, fog computing is developed and consists in a computing model that sits between IoT devices and the cloud.
It allows for the collection, aggregation, and processing of data from IoT devices using a hierarchy of computing power.
Combining fog computing with the cloud can reduce data transfers and communication bottlenecks to the cloud, and can also contribute to reduced
latencies since fog computing resources are closer to the edge.

%
\paragraph{Thesis Structure.} % Optional paragraph title
%
Accordingly, the remainder of this thesis is structured as follows.
%
\Cref{chap:background} discusses (briefly describe the content of \cref{chap:background}).
%
Describe other chapters here in a similar way.
%
Finally, \Cref{chap:conclusions} concludes this thesis by summarizing its main contribution.