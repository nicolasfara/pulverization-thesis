\chapter{\introductionname}
\label{chap:introduction}
%----------------------------------------------------------------------------------------

With the Internet of Things more and more devices are connected to the network producing a very large amount of data. Cloud computing has been
established to acquire computational power and storage for various applications; however, it is not always suitable for handling a myriad of data
from IoT devices while respecting real-time constraints.
For these reasons, fog computing is developed and consists in a computing model that sits between IoT devices and the cloud.
It allows for the collection, aggregation, and processing of data from IoT devices using a hierarchy of computing power.
Combining fog computing with the cloud can reduce data transfers and communication bottlenecks to the cloud, and can also contribute to reduced
latencies since fog computing resources are closer to the edge.

Nevertheless, realizing systems that operate in the edge-cloud continuum is an open challenge~\cite{BITTENCOURT2018134}.
The heterogeneity of devices coupled with the dynamic nature of the requirements today's systems must have, leveraging the flexibility of the
edge-cloud continuum is found to be as strategic as it is complex.

For these reasons, different approaches have been proposed to address the challenges of realizing systems that well interoperate in heterogeneous
infrastructures. Some notable methodologies and frameworks are the osmotic computing paradigm~\cite{8781958}, DR-BIP and its
extensions~\cite{10.1007/978-3-030-03424-5_20,10.1007/978-3-642-30564-1_1,de2020dream}.
While the first approach is mainly oriented to distributed microservices, the latter approach is more focused on the orchestration of distributed
applications by dynamically adapting the system to the changing requirements basing the systems on the \emph{motif} concept.

In the CPS context, engineering systems featuring distributed intelligence in a \emph{self-organization} fashion is one of the main relevant
approaches. In this way, the global behaviour of the system is obtained by the interaction of the individual components giving robustness to the
system. The current trend of large-scale, dynamic and heterogeneous Cyber-Physical Systems requires increasingly complex and diverse
infrastructures. Remote clouds offer a seemingly supply of computing, storage, and services on demand, but this comes with the caveat of high
costs and potential latency issues, as well as data protection concerns that must align with the specific requirements of each application. Edge
computing, on the other hand, brings resources closer to users, resulting in reduced latency and increased reactivity, while simultaneously
addressing data dissemination concerns.

As stated before, such infrastructures are not easy to manage and orchestrate, complicating the engineering of systems where the logic of the
systems tends to be coupled with infrastructure aspects. Generally, this prevents reusing design elements across different scenarios by exploiting the
underlying infrastructure opportunistically.

To tackle this problem the \emph{pulverization approach} is proposed~\cite{fi12110203}. This framework brakes the system behaviour into small
computational pieces that are continuously executed and scheduled in the available infrastructure. In this way, the system can be seamlessly
mapped onto a variety of multi-layered deployment infrastructures.

The main contribution of this thesis is the development of a framework that leverages the pulverization approach to orchestrate distributed
applications in the edge-cloud continuum.
This framework aims to lay the groundwork for closing the gap between the simulation of these systems and their deployment by exploiting the
pulverizing methodology.
Although the pulverization approach originates in the context of aggregate computing, the framework aims to be generic enough to enable the
pulverization in more traditional systems as well.
To showcase the effectiveness of the framework, some scenarios in the context of CPS have been identified that use the framework to deploy such
systems, highlighting the potential that sputtering possesses as a methodology.
In particular, the framework is used in conjunction with \emph{embedded systems} to recreate a heterogeneous infrastructure where the framework
runs on.

%
\paragraph{Thesis Structure.} % Optional paragraph title
%
Accordingly, the remainder of this thesis is structured as follows.
%
\Cref{chap:background} discusses the background and related works.
%
\Cref{chap:requirements} summarize the requirements of the framework and give an overview of relevant deployment scenarios that are worth 
to be considered during the validation of the framework.
%
\Cref{chap:design} presents the framework and its architectural design.
%
\Cref{chap:implementation} describes the implementation details of the framework.
%
\Cref{chap:validation} shows the validation process of the framework, including the experimental setup and the results obtained by the demos.
%
Finally, \Cref{chap:conclusions} concludes this thesis by summarizing its main contribution, focusing on future works.